%! Author = chouheiwa
%! Date = 2022/11/18

% Preamble
\documentclass[UTF8]{article} %article 文档
\usepackage{ctex}  %使用宏包(为了能够显示汉字)
\usepackage{hyperref}
\usepackage{graphicx}
\usepackage{geometry}
\geometry{a4paper,scale=0.8}
\usepackage{listings}
\usepackage{color}
\definecolor{dkgreen}{rgb}{0,0.6,0}
\definecolor{gray}{rgb}{0.5,0.5,0.5}
\definecolor{mauve}{rgb}{0.58,0,0.82}
\lstset{frame=tb,
    language=Python,
    aboveskip=3mm,
    belowskip=3mm,
    showstringspaces=false,
    columns=flexible,
    basicstyle={\small\ttfamily},
    numbers=left,%设置行号位置none不显示行号
%numberstyle=\tiny\courier, %设置行号大小
    numberstyle=\tiny\color{gray},
    keywordstyle=\color{blue},
    commentstyle=\color{dkgreen},
    stringstyle=\color{mauve},
    breaklines=true,
    breakatwhitespace=true,
    escapeinside=``,%逃逸字符(1左面的键),用于显示中文例如在代码中`中文...`
    tabsize=4,
    extendedchars=false %解决代码跨页时,章节标题,页眉等汉字不显示的问题
}

% Packages
\usepackage{amsmath}
\usepackage{amsthm}
\usepackage{amssymb}
\usepackage{wasysym}
\usepackage{float}
\usepackage{mathtools}

% new command
\newcommand{\BoldEpsilon}{\boldsymbol{\varepsilon}}
\newcommand{\BoldBeta}{\boldsymbol{\beta}}
\newcommand{\TildeBeta}{\tilde{\boldsymbol{\beta}}}
\newcommand{\HatBeta}{\hat{\boldsymbol{\beta}}}
\newcommand{\BaseX}{(\mathbf{X}^T \mathbf{X})^{-1}\mathbf{X}^T}
\newcommand{\CExpand}{\BaseX + \mathbf{D}}

% Document
\title{机器学习课程作业——问题1}
\author{chouheiwa}
\date{\today}
\linespread{1.5}
\begin{document}
    \maketitle

    \begin{proof}
        对于多元线性回归模型:
        \begin{equation}
            y_i = \sum_{j=1}^p \beta_j x_{ij} + \varepsilon_i \text{,} x_{i0} = 1 \text{,} i = 1,2,\cdots,n \label{eq:eq_y_i}
        \end{equation}

        使用矩阵可表示如下:
        \begin{equation}
            \mathbf{Y} = \mathbf{X} \mathbf{\beta} + \BoldEpsilon \label{eq:eq_y}
        \end{equation}
        其中:

        \begin{align}
            \mathbf{Y} &= \begin{bmatrix}
                              y_1    \\
                              y_2    \\
                              \vdots \\
                              y_n
            \end{bmatrix} \text{(观测值向量)} \\
            \mathbf{X} &= \begin{bmatrix}
                              1      & x_{11} & \cdots & x_{1p} \\
                              1      & x_{21} & \cdots & x_{2p} \\
                              \vdots & \vdots & \ddots & \vdots \\
                              1      & x_{n1} & \cdots & x_{np}
            \end{bmatrix} \text{(设计矩阵)} \\
            \mathbf{\beta} &= \begin{bmatrix}
                                  \beta_0 \\
                                  \beta_1 \\
                                  \vdots  \\
                                  \beta_p
            \end{bmatrix} \text{(参数向量)} \\
            \BoldEpsilon &= \begin{bmatrix}
                                \varepsilon_1 \\
                                \varepsilon_2 \\
                                \vdots        \\
                                \varepsilon_n
            \end{bmatrix} \text{(随机误差向量)}
        \end{align}

        题目可以等价于求证:

        当满足如下条件时:
        \begin{itemize}
            \item 对于任意的$\mathbf{X}$, $E(\BoldEpsilon | \mathbf{X}) = 0$ (零均值)
            \item $\mathbf{Var(\BoldEpsilon | \mathbf{X})} = E(\BoldEpsilon \BoldEpsilon^T | \mathbf{X}) = \sigma_{\mathbf{\varepsilon}}^2 I_{n}$ (方差相等且不相关)
        \end{itemize}
        最小二乘法获得的参数估计是最优的。

        令 $\TildeBeta = \mathbf{C} \mathbf{Y}$ 作为另一种线性估计器,其中 $\mathbf{C} = (\mathbf{X}^T \mathbf{X})^{-1}\mathbf{X}^T + \mathbf{D}$, $D$为非零矩阵。则有:

        \begin{align}
            E(\TildeBeta) &= E(\mathbf{C} \mathbf{Y}) \notag \\
            &= E\left[\left( \CExpand \right)(\mathbf{X} \BoldBeta+\BoldEpsilon)\right] \notag \\
            &= \left( \CExpand \right) \mathbf{X} \BoldBeta+\left( \CExpand \right) \mathrm{E}[\BoldEpsilon]
        \end{align}

        因 $\mathrm{E}[\BoldEpsilon] = 0$,则有:
        \begin{align}
            E(\TildeBeta) &= \left( \CExpand \right) \mathbf{X} \BoldBeta \notag \\
            &= \left(\left(\mathbf{X}^T \mathbf{X}\right)^{-1} \mathbf{X}^T \mathbf{X} + D\mathbf{X}\right) \BoldBeta \notag \\
            &= (\mathbf{I} + D \mathbf{X}) \BoldBeta
        \end{align}
        由于 $\BoldBeta$ 是不可知的,即 当且仅当 $D \mathbf{X} = 0$时,$\TildeBeta$ 是无偏差的。此时,偏差函数如下:
        \begin{align}
            \mathrm{Var}(\TildeBeta) &= \mathrm{Var}(\mathbf{C} \mathbf{Y}) \notag \\
            &= \mathbf{C} \mathrm{Var}(\mathbf{Y}) \mathbf{C}^T \notag \\
            &= \sigma^{2} \mathbf{C} \mathbf{C}^T \notag \\
            &= \sigma^{2} \left( \CExpand \right) \left( \mathbf{X} (\mathbf{X}^T \mathbf{X})^{-1} + D^T \right) \notag \\
            &= \sigma^{2}\left(\left(\mathbf{X}^T \mathbf{X}\right)^{-1} \mathbf{X}^T \mathbf{X}\left(\mathbf{X}^T \mathbf{X}\right)^{-1}+\left(\mathbf{X}^T \mathbf{X}\right)^{-1} \mathbf{X}^T D^{\prime}+D \mathbf{X}\left(\mathbf{X}^T \mathbf{X}\right)^{-1}+D D^T\right) \notag \\
            &= \sigma^{2}\left(\mathbf{X}^T \mathbf{X}\right)^{-1}+\sigma^{2}\left(\mathbf{X}^T \mathbf{X}\right)^{-1}(D \mathbf{X})^T+\sigma^{2} D \mathbf{X}\left(\mathbf{X}^T \mathbf{X}\right)^{-1}+\sigma^{2} D D^T
        \end{align}

        由于 $D \mathbf{X} = 0$:
        \begin{align}
            \mathrm{Var}(\TildeBeta) =  \sigma^{2}\left(\mathbf{X}^T \mathbf{X}\right)^{-1} + \sigma^{2} D D^T
        \end{align}

        又有 $\sigma^{2} (\mathbf{X}^T \mathbf{X}) = \mathrm{Var}(\HatBeta)$,$\HatBeta$为最小二乘法的估计参数。由此可得:
        \begin{align}
            \mathrm{Var}(\TildeBeta) = \mathrm{Var}(\HatBeta) + \sigma^{2} D D^T
        \end{align}

        因为 $D D^T$ 为正定矩阵,所以 $\mathrm{Var}(\TildeBeta) \geq \mathrm{Var}(\HatBeta)$。即对于其他的任意线性估计器,其偏差均不小于最小二乘法的方差。由此可证,最小二乘法是最优的。
    \end{proof}

    注: 本题等价于证明 高斯-马尔可夫定理~\cite{wiki},即任意线性估计器的方差均不小于最小二乘法的方差。
    \bibliography{question1}
    \bibliographystyle{plain}
\end{document}