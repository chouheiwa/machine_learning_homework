%! Author = chouheiwa
%! Date = 2022/11/05

% Preamble
\documentclass[UTF8]{article} %article 文档
\usepackage{ctex}  %使用宏包(为了能够显示汉字)
\usepackage{hyperref}
\usepackage{graphicx}
\usepackage{geometry}
\geometry{a4paper,scale=0.8}
\usepackage{listings}
\usepackage{color}
\definecolor{dkgreen}{rgb}{0,0.6,0}
\definecolor{gray}{rgb}{0.5,0.5,0.5}
\definecolor{mauve}{rgb}{0.58,0,0.82}
\lstset{frame=tb,
    language=Python,
    aboveskip=3mm,
    belowskip=3mm,
    showstringspaces=false,
    columns=flexible,
    basicstyle={\small\ttfamily},
    numbers=left,%设置行号位置none不显示行号
%numberstyle=\tiny\courier, %设置行号大小
    numberstyle=\tiny\color{gray},
    keywordstyle=\color{blue},
    commentstyle=\color{dkgreen},
    stringstyle=\color{mauve},
    breaklines=true,
    breakatwhitespace=true,
    escapeinside=``,%逃逸字符(1左面的键),用于显示中文例如在代码中`中文...`
    tabsize=4,
    extendedchars=false %解决代码跨页时,章节标题,页眉等汉字不显示的问题
}

% Packages
\usepackage{amsmath}
\usepackage{amsthm}
\usepackage{amssymb}
\usepackage{wasysym}

% Document
\title{机器学习课程作业-问题2}
\author{chouheiwa}
\date{2022/11/05}
\linespread{1.5}
\begin{document}
    \maketitle
    \tableofcontents
    因绘图与矩阵计算,因此引入了numpy和matplotlib两个包,同时由于需要解析mat文件,因此引入了scipy包。


    \section{第一问}
    最近邻算法其实并没有训练过程,总体思路是获取训练集中的每一个样本,然后计算测试样本与训练样本的距离,最后选择距离最近的k个训练样本中的众数作为测试样本预测结果。

    knn算法实现文件为~\href{run:knn_classifier.py}{knn\_classifier.py}。

    使用scipy.io.loadmat函数加载mat文件,然后将数据集,并将数据集展开做标注,随后尝试使用$k \in 1 , 2,\dots, 100$,并将测试集数据代入knn算法中,最后绘制出$k$与准确率的关系图,如下:

    \begin{figure}[htbp]
        \centering
        \includegraphics[width=0.8\textwidth]{../knn_accuracy}
        \caption{最近邻算法在预测集的准确率随k值变化图}
    \end{figure}

    可以发现当$k > 40$后,随着k的增长,其准确率下降的趋势越来越明显,因此可以判定这里无需增加$k$的取值范围了,当前准确率即可找到最优$k_{opt}$

    从上述准确率中获取最大的准确率为$A = 0.976667$,其$k = 14$。

    即答案$k_{opt} = 14$, 对应的错误率为 $E_{opt} = 1 - A_{opt} = 0.023333$


    \section{第二问}

    从上一问数据中同样可以获得:

    当$k = 1$时,对应的$A_{k=1} = 0.966667$

    当$k = 50$时, 对应的$A_{k=50} = 0.968667$

    这里上面两种的正确率均比$A_{k_{opt} = 14} = 0.976667$ 有一定程度的下降。

    当$k = 1$时,这里算法发生了过拟合现象,即其与训练集算法准确率达到$1$,而对于测试集,因为只受到最近的一个点的影响,所以很容易收到训练集的噪点影响,所以正确率会发生下降

    当$k = 50$时,算法将会发生欠拟合,从极限角度考虑,我们取$k = 1500$时,此时不论我们使用任何点,参与运算,其与训练集的多种类别数量均为500,此时,算法将无法给出正确答案。

    当$k = 14$时,此时属于最优情况,对于测试集,其准确率为最大。


    \section{第三问}

    似然率分类器的处理过程如下:
    \begin{enumerate}
        \item 对于每一个类别,计算其样本集中均值向量$\mu_{i}$,协方差矩阵$\Sigma_{i}$,因假设其先验概率相等,因此$P_{i} = \frac{1}{3}$
        \item 对于测试集中的每一个样本,计算其对应的每一个分布的,并应用贝叶斯公式,计算其对应的后验概率,取最大的后验概率对应的类别作为其预测类别
    \end{enumerate}

    似然分类器的分类准确率为: $A_{likelihood} = 0.771333$

    最近邻算法的分类准确率为: $A_{k_{opt}} = 0.976667$

    可以看出,最近邻算法的分类准确率要远远高于似然分类器,因此最近邻算法的分类效果要好于似然分类器。

    对应分类界面为:

    \begin{figure}[htbp]
        \begin{minipage}[t]{0.5\linewidth}
            \centering
            \includegraphics[width=\linewidth]{../likelihood_classifier}
            \caption{似然分类器分类界面}
        \end{minipage}%
        \begin{minipage}[t]{0.5\linewidth}
            \centering
            \includegraphics[width=\linewidth]{../knn_decision_boundary}
            \caption{最近邻算法分类界面}
        \end{minipage}
    \end{figure}

    可以看出,最近邻算法的分类界面更适配于数据集,而似然分类器的分类界面则不太适配。

\end{document}