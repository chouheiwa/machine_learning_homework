%! Author = chouheiwa
%! Date = 2022/10/21

% Preamble
\documentclass[UTF8]{article} %article 文档
\usepackage{ctex}  %使用宏包(为了能够显示汉字)
\usepackage{hyperref}
\usepackage{graphicx}
\usepackage{geometry}
\geometry{a4paper,scale=0.8}
\usepackage{listings}
\usepackage{color}
\definecolor{dkgreen}{rgb}{0,0.6,0}
\definecolor{gray}{rgb}{0.5,0.5,0.5}
\definecolor{mauve}{rgb}{0.58,0,0.82}
\lstset{frame=tb,
    language=Python,
    aboveskip=3mm,
    belowskip=3mm,
    showstringspaces=false,
    columns=flexible,
    basicstyle={\small\ttfamily},
    numbers=left,%设置行号位置none不显示行号
%numberstyle=\tiny\courier, %设置行号大小
    numberstyle=\tiny\color{gray},
    keywordstyle=\color{blue},
    commentstyle=\color{dkgreen},
    stringstyle=\color{mauve},
    breaklines=true,
    breakatwhitespace=true,
    escapeinside=``,%逃逸字符(1左面的键),用于显示中文例如在代码中`中文...`
    tabsize=4,
    extendedchars=false %解决代码跨页时,章节标题,页眉等汉字不显示的问题
}

% Packages
\usepackage{amsmath}

% Document
\title{机器学习课程作业1}
\author{chouheiwa}
\date{2022/10/21}
\linespread{1.5}
\begin{document}
    \maketitle


    \section{问题一}
    这里无法避免使用到矩阵的算法,以及绘图的算法,所以代码中会额外需要使用到numpy(用以进行数学运算)和matplotlib(用以生成和绘制相关散点图及分界线图)这两个库。

    \subsection{第一小问}
    这里已知要生成N = 1000个二维样本的数据集$X_1$和$X_2$。又知晓样本来自于三个不同的正态分布,其分布的均值矢量分别为
    \[
        m_1 = [1, 1]^T\\m_2 = [4, 4]^T\\m_3 = [8, 1]^T
    \]
    协方差为
    \[
        \Sigma_1 = \Sigma_2 = \Sigma_3 = \begin{bmatrix}
                                             2 & 0 \\ 0 & 2
        \end{bmatrix}
    \]

    因此,我们可以先生成三个正态分布的随机数,然后再将其乘以协方差矩阵,最后再加上均值矢量,即可得到三个正态分布的样本。这里我们使用 numpy 的 random.multivariate\_normal 函数来生成对应数据集。

    生成数据集功能使用了名为GenerateData作为根据给定参数自动生成相关数据集的功能的类,其代码于文件\href{run:generate_data.py}{generate\_data.py}中。

    绘制数据图像功能代码于文件\href{run:plot_image.py}{plot\_image.py}中。

    main.py函数中会调用上述部分代码,生成数据集并绘制对应数据集$X_1$、$X_2$的散点图图像。示例结果如下图所示。

    \begin{figure}[htbp]
        \begin{minipage}[t]{0.45\linewidth}
            \centering
            \includegraphics[height=4cm,width=4cm]{../question1/X1_data}
        \end{minipage}%
        \begin{minipage}[t]{0.45\linewidth}
            \centering
            \includegraphics[height=4cm,width=4cm]{../question1/X2_data}
        \end{minipage}\label{fig:figure}
    \end{figure}

    \subsection{第二小问}
    代码总体解决方案上因为采用了父类和子类的多态设计,所以在main.py中只需要调用类的方法即可,不需要关心具体的实现细节。因此,这里只需要介绍类的设计思路。

    类文件为\href{run:probability_calculate.py}{probability\_calculate.py}。其中,父类(基类)为 ProbabilityCalculate 其主要负责实现通用判别函数,以及其他相关绘制分类线的方法,父类的存在使得子类只需要继承父类的功能,并实现对应计算$g_k(x)$函数的方法即可作为对应的分类器,而无需关心其他细节。加强了代码的复用性。

    因此这里对接下来的三种分类器均计算出$g_k(x)$的表达式便可以完成对应的分类器的设计。

    \subsubsection{似然率决策规则}\label{subsubsec:likelihood}
    似然率决策规则的基本思想是,对于给定的样本,我们可以通过计算其属于各个类别的概率,然后选择概率最大的类别作为其所属类别。因此,对于给定的样本$x$,其属于类别$m_k$的概率为
    \begin{equation}
        P(m_k|x) = \frac{P(x|m_k)P(m_k)}{P(x)} \label{eq:a}
    \end{equation}

    其中,$P(x|m_k)$为似然率,$P(m_k)$为先验概率,$P(x)$为归一化因子。由于$P(x)$对于所有类别$k$都是相同的,因此我们可以忽略该项,即
    \begin{equation}
        P(m_k|x) \propto P(x|m_k)P(m_k) \label{eq:b}
    \end{equation}

    即可以令
    \begin{equation}
        g_k(x) = P(m_k|x) \label{eq:c}
    \end{equation}
    则我们可以通过计算$g_k(x)$的值,然后选择最大的$g_k(x)$对应的类别$m_k$作为样本$x$的预测类别。

    又因为题目中给定数据集是从属于三个正态分布的$m_1$、$m_2$、$m_3$,所以其属于类别$m_k$的似然率$P(x|m_k)$可以写成:
    \begin{equation}
        P(x|m_k) = \frac{1}{(2\pi)^{\frac{d}{2}}|\Sigma_k|^{\frac{1}{2}}}\exp\left(-\frac{1}{2}(x-m_k)^T\Sigma_k^{-1}(x-m_k)\right) \label{eq:e}
    \end{equation}
    将~\eqref {eq:e}代入~\eqref {eq:b},则有
    \begin{equation}
        P(m_k|x) \propto \frac{1}{(2\pi)^{\frac{d}{2}}|\Sigma_k|^{\frac{1}{2}}}\exp\left(-\frac{1}{2}(x-m_k)^T\Sigma_k^{-1}(x-m_k)\right)P(m_k) \label{eq:f}
    \end{equation}
    根据题目可知,$m_1$、$m_2$、$m_3$所属正态分布的协方差矩阵均相同,即$|\Sigma_k|$为常数,此时我们可以继续忽略常数项,便可由~\eqref {eq:f}得到
    \begin{equation}
        P(m_k|x) \propto \exp\left(-\frac{1}{2}(x-m_k)^T\Sigma_k^{-1}(x-m_k)\right)P(m_k) \label{eq:g}
    \end{equation}
    此时,我们可以将~\eqref {eq:g}代入~\eqref {eq:c},则有
    \begin{equation}
        g_k(x) = \exp\left(-\frac{1}{2}(x-m_k)^T\Sigma_k^{-1}(x-m_k)\right)P(m_k) \label{eq:h}
    \end{equation}
    似然率决策规则的分类器设计实验完毕。其代码所属类名为 LikelihoodProbability。

    \subsubsection{贝叶斯风险决策规则}
    贝叶斯风险决策规则的基本思想是,对于给定的样本,我们可以通过计算其属于各个类别的风险,然后选择风险最小的类别作为其所属类别。因此,对于给定的样本$x$,其属于类别$m_i$的风险为
    \begin{equation}
        \Re(\alpha_i|x) = = \sum_{j = 1}^{C} C_{ij} P(m_j|x) \label{eq:i}
    \end{equation}

    其后验概率 $P(m_j|x)$ 表达式为

    \begin{equation}
        P(m_j|x) = \frac{1}{(2\pi)^{\frac{d}{2}} \sqrt{|\Sigma_j|}}e^{-\frac{1}{2}(x-m_j)^T\Sigma_j^{-1}(x-m_j)}\frac{P(m_j)}{p(x)} \label{eq:j}
    \end{equation}

    与似然决策规则相同,其中$\frac{1}{(2\pi)^{\frac{d}{2}} \sqrt{|\Sigma_j|}}$ 与 $\frac{1}{p(x)}$ 为常数,因此可以忽略不计。则~\eqref {eq:j}可写成:

    \begin{equation}
        P(m_j|x) \propto \exp\left( -\frac{1}{2}(x-m_j)^T\Sigma_k^{-1}(x-m_j) \right) P(m_j) \label{eq:k}
    \end{equation}

    贝叶斯风险决策的$g_i(x)$等于~\eqref {eq:i} 同时将~\eqref {eq:k} 代入~\eqref {eq:i},则有:
    \begin{equation}
        g_i(x) = \sum_{j = 1}^{C} C_{ij} \exp\left( -\frac{1}{2}(x-m_j)^T\Sigma_j^{-1}(x-m_j) \right) P(m_j) \label{eq:l}
    \end{equation}

    与似然率决策规则~\ref{subsubsec:likelihood}不同,贝叶斯风险决策规则需要选取风险最小即$g_i(x)$最小的类别作为其所属类别。

    贝叶斯风险决策规则的分类器设计实验完毕。其代码所属类名为 BayesProbability。

    \subsubsection{最小欧几里得距离分类器}
    其$g_i(x) = \frac{1}{2}(x-m_i)^T(x-m_i)$,去除常数项后为$g_i(x) =  -(x-m_i)^T(x-m_i)$

    最小欧几里得距离分类器的分类器设计实验完毕。其代码所属类名为 EuclidProbability。

    \subsection{第三小问 实验结果分析和分类决策界面与错误}
    实验代码如\href{run:main.py}{main.py}中的主函数部分所示。实验结果如下图所示。

\end{document}